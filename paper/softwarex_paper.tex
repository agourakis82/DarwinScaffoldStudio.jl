\documentclass[final,5p,times,twocolumn]{elsarticle}

\usepackage{hyperref}
\usepackage{graphicx}
\usepackage{booktabs}
\usepackage{siunitx}
\usepackage{xcolor}
\usepackage{listings}

\lstset{
  language=Julia,
  basicstyle=\ttfamily\scriptsize,
  keywordstyle=\color{blue}\bfseries,
  commentstyle=\color{gray},
  breaklines=true,
  frame=single
}

\journal{SoftwareX}

\begin{document}

\begin{frontmatter}

\title{Darwin Scaffold Studio: Scaffold morphometry with ontology-aware metadata for FAIR tissue engineering research}

\author[pucsp]{Demetrios Chiuratto Agourakis\corref{cor1}}
\ead{demetrios@agourakis.med.br}
\cortext[cor1]{Corresponding author}

\author[pucsp]{Moema Alencar Hausen}

\affiliation[pucsp]{organization={Pontifical Catholic University of S\~ao Paulo},
            city={S\~ao Paulo}, country={Brazil}}

\begin{abstract}
Scaffold pore size measurement lacks methodological consensus: different techniques yield incompatible values, and no gold standard exists for validation. Darwin Scaffold Studio addresses this challenge through transparent algorithm implementation with documented error characteristics. The software computes porosity, pore size distribution (local thickness), interconnectivity, and geometric tortuosity from microCT and SEM images. Uniquely, Darwin integrates 1,200+ terms from OBO Foundry biomedical ontologies (UBERON, CL, CHEBI), enabling standardized, machine-readable scaffold characterization. Validation against TPMS analytical surfaces shows $<$1\% porosity error. Validation against the PoreScript dataset yields 14.1\% pore size error---higher than PoreScript's 5\% but with fully documented bias, enabling users to apply corrections. Darwin's contribution is not superior accuracy, but rather the combination of validated morphometry with ontology-aware FAIR data export, addressing the reproducibility crisis in scaffold characterization.
\end{abstract}

\begin{keyword}
scaffold characterization \sep pore size measurement \sep biomedical ontology \sep FAIR data \sep tissue engineering \sep open source
\end{keyword}

\end{frontmatter}

%% METADATA TABLE
\section*{Metadata}

\begin{table}[h]
\small
\begin{tabular}{p{0.35\linewidth}p{0.55\linewidth}}
\toprule
\textbf{Nr} & \textbf{Code metadata} \\
\midrule
C1 & v0.3.0 \\
C2 & \url{https://github.com/agourakis82/darwin-scaffold-studio} \\
C3 & \url{https://doi.org/10.5281/zenodo.17832882} \\
C4 & MIT License \\
C5 & git \\
C6 & Julia 1.10+ \\
C7 & Images.jl, Statistics.jl, XLSX.jl \\
C8 & README.md, docs/, examples/ \\
C9 & demetrios@agourakis.med.br \\
\bottomrule
\end{tabular}
\end{table}

%% ============================================
\section{Motivation and significance}
%% ============================================

Scaffold pore size is a critical parameter in tissue engineering: Murphy et al. demonstrated that 100--200~\si{\micro\meter} pores optimize osteoblast attachment \cite{murphy2010}, while Karageorgiou and Kaplan established porosity $>$90\% as prerequisite for bone ingrowth \cite{karageorgiou2005}. Despite this importance, pore size measurement remains problematic.

Loh and Choong observed that ``to date, no agreement has been found with respect to the methodology for pore size evaluation'' \cite{loh2013}. Different techniques measure fundamentally different quantities: SEM captures 2D cross-sections subject to orientation bias; micro-CT provides 3D data but with resolution limitations; mercury porosimetry measures constriction points rather than pore volumes. Paxton et al. noted that ``since the determination of the exact pore size value is not possible, the comparison of the various methods applied is complicated'' \cite{paxton2018}.

This methodological ambiguity creates a reproducibility problem. When Study A reports 150~\si{\micro\meter} pores using SEM and Study B reports 200~\si{\micro\meter} using micro-CT for ostensibly similar scaffolds, researchers cannot determine whether the difference reflects material variation or measurement methodology.

Darwin Scaffold Studio addresses this challenge through two contributions:

\textbf{Transparent error characterization.} Rather than claiming accuracy, Darwin documents its systematic biases. Our validation shows 14.1\% underestimation on SEM images---higher than PoreScript's 5\% \cite{porescript}---but with characterized error enabling user correction.

\textbf{Ontology-aware metadata.} Darwin integrates 1,200+ terms from OBO Foundry ontologies, enabling standardized description of scaffold properties. When a scaffold is annotated with UBERON:0002481 (bone tissue) and CHEBI:53310 (polycaprolactone), the measurement methodology, error characteristics, and material properties become machine-queryable, facilitating meta-analysis across studies.

The software serves researchers who prioritize reproducibility and interoperability over marginal accuracy improvements.

%% ============================================
\section{Software description}
%% ============================================

\subsection{Architecture}

Darwin implements a modular architecture separating image processing, morphometric algorithms, and semantic annotation:

\begin{itemize}
\item \textbf{MicroCT module}: Image I/O (RAW, TIFF, NIfTI), Otsu segmentation, binary volume operations
\item \textbf{Metrics module}: Porosity, local thickness, interconnectivity, Dijkstra tortuosity
\item \textbf{Ontology module}: OBO Foundry integration, 3-tier lookup, JSON-LD export
\item \textbf{Validation module}: TPMS surface generation, ground truth comparison
\end{itemize}

\subsection{Morphometric algorithms}

\textbf{Porosity.} Computed by voxel counting: $\phi = V_{\text{pore}} / V_{\text{total}}$. Mathematically exact for binary volumes.

\textbf{Pore size.} Implements Hildebrand-R\"uegsegger local thickness \cite{hildebrand1997}: each pore voxel receives the diameter of the largest inscribed sphere containing it. This provides a distribution capturing pore size heterogeneity, rather than a single mean that obscures structural variation.

\textbf{Interconnectivity.} Ratio of largest connected pore component to total pore volume (26-connectivity). Values $>$90\% indicate suitable percolation for tissue ingrowth.

\textbf{Tortuosity.} Geometric path analysis using Dijkstra's algorithm:
\begin{equation}
\tau = L_{\text{geodesic}} / L_{\text{Euclidean}}
\end{equation}
Unlike empirical approximations ($\tau \approx \phi^{-0.5}$), this yields directional tortuosity ($\tau_x, \tau_y, \tau_z$) for anisotropic scaffolds.

\subsection{Ontology integration}

Darwin integrates terms from seven OBO Foundry ontologies:

\begin{itemize}
\item \textbf{UBERON}: 20+ tissue types (bone, cartilage, skin) with literature-derived optimal parameters
\item \textbf{CL}: 20+ cell types (osteoblast, MSC, chondrocyte) with size and marker specifications
\item \textbf{CHEBI}: 25+ biomaterials (PCL, PLA, HA) with CAS numbers and mechanical properties
\item \textbf{GO, NCIT, BTO, DOID}: Biological processes, diseases, cell lines
\end{itemize}

A 3-tier lookup system provides: (1) 150 hardcoded core terms for offline use, (2) 5,000 cached terms from local OWL files, (3) online API access (EBI OLS, NCBO BioPortal) with SQLite caching.

Export uses JSON-LD with Schema.org vocabulary and persistent URIs, ensuring FAIR compliance: scaffolds become findable by ontology term, accessible via standard formats, interoperable across systems, and reusable with explicit provenance.

%% ============================================
\section{Illustrative examples}
%% ============================================

\subsection{Basic scaffold analysis}

\begin{lstlisting}[caption={Scaffold characterization with ontology annotation}]
# Load SEM image
img = load("scaffold.tif")
binary = segment_otsu(img)

# Compute metrics
metrics = compute_metrics(binary,
                          pixel_size=3.5)
# porosity: 0.72
# pore_size: 165 +/- 48 um
# interconnectivity: 0.98

# Export with semantic annotation
export_fair("scaffold.jsonld", metrics,
  tissue = "UBERON:0002481",  # bone
  material = "CHEBI:53310",   # PCL
  method = "SEM_connected_components",
  error_estimate = 0.15)  # 15% bias
\end{lstlisting}

\subsection{Validation workflow}

\begin{lstlisting}[caption={Validation against analytical ground truth}]
# Generate TPMS with known porosity
gyroid = generate_tpms(:gyroid,
                       target_porosity=0.70)

# Compute and compare
measured = compute_porosity(gyroid)
error = abs(measured - 0.70) / 0.70
# error: 0.001 (<1%)
\end{lstlisting}

%% ============================================
\section{Impact}
%% ============================================

\subsection{Validation results}

\textbf{Analytical ground truth.} Validation against TPMS surfaces (Gyroid, Schwarz P/D, Neovius) with known geometry shows $<$1\% porosity error across 16 test cases. Interconnectivity exceeds 99\% for all TPMS types, confirming algorithmic correctness.

\textbf{Experimental validation.} We validated against the PoreScript dataset \cite{porescript} (DOI: 10.5281/zenodo.5562953), which provides SEM images of salt-leached scaffolds with 374 manual measurements across 3 images.

\begin{table}[h]
\centering
\caption{Validation: Darwin vs. manual measurements}
\label{tab:validation}
\small
\begin{tabular}{lccc}
\toprule
Sample & Darwin & Manual & APE \\
\midrule
S1\_27x & 143~\si{\micro\meter} & 170~\si{\micro\meter} & 15.8\% \\
S2\_27x & 152~\si{\micro\meter} & 176~\si{\micro\meter} & 13.4\% \\
S3\_27x & 154~\si{\micro\meter} & 177~\si{\micro\meter} & 12.9\% \\
\midrule
\textbf{Mean} & 149~\si{\micro\meter} & 174~\si{\micro\meter} & \textbf{14.1\%} \\
\bottomrule
\end{tabular}
\end{table}

Darwin systematically underestimates pore size by $\sim$15\%. This is inferior to PoreScript's $\sim$5\% difference \cite{porescript}. However, Darwin's error is \emph{consistent and documented}, enabling users to apply correction factors when absolute values are required. For comparative studies (scaffold A vs. B), systematic bias cancels.

\subsection{Comparison with existing tools}

\begin{table}[h]
\centering
\caption{Feature comparison}
\label{tab:comparison}
\small
\begin{tabular}{lccc}
\toprule
 & Darwin & BoneJ & PoreScript \\
\midrule
Open source & Yes & Yes & Yes \\
3D analysis & Yes & Yes & No \\
Local thickness & Yes & Yes & No \\
Geometric $\tau$ & Yes & No & No \\
Ontology & \textbf{Yes} & No & No \\
FAIR export & \textbf{Yes} & No & No \\
Documented error & \textbf{Yes} & No & Yes \\
\bottomrule
\end{tabular}
\end{table}

Darwin's unique contribution is the combination of 3D morphometry with ontology-aware metadata export. BoneJ provides superior local thickness implementation but lacks semantic annotation. PoreScript achieves better SEM accuracy but is limited to 2D analysis.

\subsection{Addressing the reproducibility problem}

The scaffold characterization literature suffers from incomparable measurements: different labs use different methods, terminology, and unreported parameters. Darwin addresses this through:

\begin{enumerate}
\item \textbf{Explicit methodology}: Algorithm parameters are recorded in exported metadata
\item \textbf{Standardized terminology}: OBO ontology terms replace ambiguous descriptions
\item \textbf{Error documentation}: Known biases are quantified rather than hidden
\item \textbf{Machine readability}: JSON-LD enables programmatic meta-analysis
\end{enumerate}

\subsection{Limitations}

\begin{itemize}
\item Validation limited to 3 SEM images (n=374 measurements)
\item 14.1\% error exceeds PoreScript's 5\%
\item No GUI---command-line interface requires Julia familiarity
\item Ontology coverage incomplete for some biomaterial classes
\end{itemize}

%% ============================================
\section{Conclusions}
%% ============================================

Darwin Scaffold Studio provides scaffold morphometry with ontology-aware FAIR data export. The software does not claim superior accuracy---our 14.1\% pore size error exceeds alternatives. Rather, Darwin's contribution is the combination of validated algorithms with standardized semantic annotation, addressing the reproducibility crisis in scaffold characterization.

Future development includes: cross-validation with BoneJ on identical datasets, expanded ontology coverage, and web-based interface for broader accessibility.

\section*{Acknowledgments}

The authors thank the PUC-SP Biomaterials and Regenerative Medicine Program.

\section*{Declaration of competing interest}

The authors declare no competing interests.

\bibliographystyle{elsarticle-num}
\begin{thebibliography}{10}

\bibitem{murphy2010}
C.M. Murphy, M.G. Haugh, F.J. O'Brien, The effect of mean pore size on cell attachment, proliferation and migration in collagen-glycosaminoglycan scaffolds for bone tissue engineering, Biomaterials 31 (2010) 461--466.

\bibitem{karageorgiou2005}
V. Karageorgiou, D. Kaplan, Porosity of 3D biomaterial scaffolds and osteogenesis, Biomaterials 26 (2005) 5474--5491.

\bibitem{loh2013}
Q.L. Loh, C. Choong, Three-dimensional scaffolds for tissue engineering applications: role of porosity and pore size, Tissue Eng. Part B 19 (2013) 485--502.

\bibitem{paxton2018}
N.C. Paxton, et al., Note on the use of different approaches to determine the pore sizes of tissue engineering scaffolds: what do we measure?, Biomed. Eng. Online 17 (2018) 110.

\bibitem{hildebrand1997}
T. Hildebrand, P. R\"uegsegger, A new method for the model-independent assessment of thickness in three-dimensional images, J. Microsc. 185 (1997) 67--75.

\bibitem{porescript}
L.E. Wistlich, et al., PoreScript: Semi-automated pore size algorithm for scaffold characterization, J. Biomed. Mater. Res. A 110 (2022) 1061--1071.

\end{thebibliography}

\end{document}
