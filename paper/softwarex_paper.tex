\documentclass[authoryear,preprint,review,12pt]{elsarticle}

\usepackage{hyperref}
\usepackage{graphicx}
\usepackage{booktabs}
\usepackage{listings}
\usepackage{xcolor}
\usepackage{amsmath}
\usepackage{siunitx}

% Code listing style
\lstset{
  language=Julia,
  basicstyle=\ttfamily\footnotesize,
  keywordstyle=\color{blue}\bfseries,
  commentstyle=\color{gray}\itshape,
  stringstyle=\color{red},
  numbers=left,
  numberstyle=\tiny\color{gray},
  frame=single,
  breaklines=true,
  captionpos=b
}

\journal{SoftwareX}

\begin{document}

\begin{frontmatter}

\title{Darwin Scaffold Studio: An Open-Source Julia Platform for Validated, Ontology-Aware Tissue Engineering Scaffold Analysis}

\author[pucsp]{Demetrios Chiuratto Agourakis\corref{cor1}}
\ead{demetrios@agourakis.med.br}
\cortext[cor1]{Corresponding author}

\author[pucsp]{Moema Alencar Hausen}

\affiliation[pucsp]{organization={Biomaterials and Regenerative Medicine Program, Pontifical Catholic University of S\~ao Paulo (PUC-SP)},
            city={S\~ao Paulo},
            country={Brazil}}

\begin{abstract}
Darwin Scaffold Studio is an open-source Julia platform that addresses three critical gaps in tissue engineering scaffold analysis: (1) lack of validation against Q1 literature, (2) absence of standardized biomedical terminology, and (3) limited algorithmic rigor in morphometric analysis. The platform implements the Hildebrand-R\"uegsegger local thickness algorithm for pore size distribution and Dijkstra-based geometric tortuosity with directional analysis. It integrates 1,200+ terms from seven OBO Foundry ontologies (UBERON, CL, CHEBI, GO, NCIT, BTO, DOID), enabling FAIR-compliant scaffold characterization. Validation against analytical TPMS surfaces achieves 0\% error for porosity and interconnectivity. Validation against the PoreScript dataset with manual SEM measurements yields 14.1\% absolute percentage error on pore size, with systematic underestimation documented for transparency. Darwin Scaffold Studio provides the first scaffold analysis tool with public Q1 literature validation, standardized ontology integration, and honest error reporting.
\end{abstract}

\begin{keyword}
tissue engineering \sep scaffold analysis \sep local thickness \sep geometric tortuosity \sep OBO Foundry \sep biomedical ontology \sep FAIR data \sep Julia
\end{keyword}

\end{frontmatter}

%% ============================================
\section{Motivation and significance}
%% ============================================

Tissue engineering scaffolds serve as three-dimensional templates for cell attachment, proliferation, and tissue regeneration. The structural properties of scaffolds---porosity, pore size, interconnectivity, and tortuosity---directly determine biological outcomes \cite{murphy2010,karageorgiou2005}. Murphy et al. demonstrated that mean pore size between 100--200~\si{\micro\meter} optimizes osteoblast attachment in collagen-GAG scaffolds, while Karageorgiou and Kaplan established that porosity exceeding 90\% and interconnectivity above 90\% are prerequisites for bone ingrowth.

Despite the importance of accurate scaffold characterization, current analysis tools exhibit three critical limitations:

\textbf{Lack of validation.} Commercial software (CTAn, Avizo) and open-source alternatives (BoneJ) rarely publish validation against analytical ground truth or experimental measurements. Researchers cannot assess measurement accuracy or systematic biases.

\textbf{Absence of standardized terminology.} Scaffold properties are reported using inconsistent terminology. ``Porosity'' may refer to total, open, or closed porosity. ``Pore size'' may indicate mean diameter, equivalent spherical diameter, or maximum inscribed sphere. This ambiguity impedes meta-analysis and reproducibility.

\textbf{Algorithmic limitations.} Many tools approximate pore size using simple connectivity analysis rather than rigorous local thickness methods \cite{hildebrand1997}. Tortuosity is often estimated using empirical correlations (e.g., Gibson-Ashby: $\tau \approx \phi^{-0.5}$) rather than geometric path analysis.

Darwin Scaffold Studio addresses these gaps through three innovations: (1) public validation against Q1 literature with quantified error margins, (2) integration of 1,200+ terms from OBO Foundry biomedical ontologies for standardized, machine-readable characterization, and (3) implementation of rigorous algorithms including Hildebrand-R\"uegsegger local thickness and Dijkstra-based directional tortuosity.

The platform is implemented in Julia 1.10+ for high performance and reproducibility. It is open-source (MIT license) and available at \url{https://doi.org/10.5281/zenodo.17832882}.

%% ============================================
\section{Software description}
%% ============================================

\subsection{Software architecture}

Darwin Scaffold Studio employs a layered architecture separating data structures, image processing, semantic annotation, and validation (Figure~\ref{fig:arch}):

\begin{itemize}
\item \textbf{Core}: Type definitions, configuration, error handling
\item \textbf{MicroCT}: Image loading (RAW, TIFF, NIfTI), preprocessing, segmentation, morphometric analysis
\item \textbf{Ontology}: OBO Foundry integration, 3-tier lookup system, FAIR export
\item \textbf{Science}: Topology analysis, percolation theory, mechanical property prediction
\item \textbf{Validation}: Q1 literature database, TPMS ground truth generation
\end{itemize}

\subsection{Morphometric algorithms}

\subsubsection{Local thickness for pore size distribution}

Pore size is computed using the local thickness algorithm of Hildebrand and R\"uegsegger \cite{hildebrand1997}, which assigns to each pore voxel the diameter of the largest inscribed sphere containing that voxel. This provides a physically meaningful pore size distribution rather than a single mean value.

The algorithm proceeds in three steps:
\begin{enumerate}
\item Compute Euclidean distance transform $D(x)$ for all pore voxels
\item Identify local maxima of $D(x)$ as sphere centers
\item Propagate maximum inscribed sphere diameter to all contained voxels
\end{enumerate}

The implementation provides both exact (full 3D Euclidean) and fast (separable Saito-Toriwaki passes) variants, with the fast variant achieving $O(n)$ complexity for $n$ voxels.

\subsubsection{Geometric tortuosity via shortest path analysis}

Tortuosity quantifies the effective path length through the pore network relative to the straight-line distance. Darwin implements geometric tortuosity using Dijkstra's algorithm with 26-connectivity:

\begin{equation}
\tau = \frac{L_{\text{geodesic}}}{L_{\text{Euclidean}}}
\end{equation}

where $L_{\text{geodesic}}$ is the shortest path through connected pore space. Unlike empirical approximations, this approach yields directional tortuosity ($\tau_x$, $\tau_y$, $\tau_z$), enabling characterization of anisotropic scaffolds.

For large volumes, a random walk simulation provides approximate tortuosity with reduced computational cost.

\subsubsection{Additional metrics}

\begin{itemize}
\item \textbf{Porosity}: Voxel counting (mathematically exact for binary volumes)
\item \textbf{Interconnectivity}: Ratio of largest connected pore component to total pore volume (26-connectivity)
\item \textbf{Specific surface area}: Marching cubes triangulation
\item \textbf{Mechanical properties}: Gibson-Ashby scaling ($E/E_s = C(\rho/\rho_s)^n$) \cite{gibson1997}
\item \textbf{Permeability}: Kozeny-Carman relation
\end{itemize}

\subsection{OBO Foundry ontology integration}

Darwin integrates terms from seven OBO Foundry ontologies to provide standardized, interoperable scaffold characterization:

\begin{itemize}
\item \textbf{UBERON} (anatomy): 20+ tissue types with optimal scaffold parameters
\item \textbf{CL} (cell types): 20+ cell types including MSC, osteoblast, chondrocyte
\item \textbf{CHEBI} (materials): 25+ biomaterials with CAS numbers and mechanical properties
\item \textbf{GO} (biological process): Ossification, angiogenesis, wound healing
\item \textbf{NCIT} (disease): Tissue defects requiring scaffold intervention
\item \textbf{BTO} (cell lines): Standard cell lines for in vitro testing
\item \textbf{DOID} (disease): Cross-linked disease classifications
\end{itemize}

The ontology system employs a 3-tier lookup architecture:
\begin{enumerate}
\item \textbf{Tier 1}: 150 hardcoded core terms for instant offline access
\item \textbf{Tier 2}: 5,000 extended terms lazy-loaded from local OWL cache
\item \textbf{Tier 3}: Online API (EBI OLS, NCBO BioPortal) with SQLite caching
\end{enumerate}

This enables queries such as:
\begin{lstlisting}[caption={Ontology-guided scaffold design}]
# Lookup optimal parameters for bone tissue
bone = lookup_term(:bone, :tissue)  # UBERON:0002481
cells = get_cells_for_tissue(bone)  # osteoblast, MSC, osteocyte
materials = get_materials_for_tissue(bone)  # HA, TCP, PCL, collagen
\end{lstlisting}

All exported data includes persistent URIs (PURLs) and Schema.org vocabulary for FAIR compliance.

\subsection{Q1 literature validation framework}

Darwin includes a validation framework referencing nine Q1 publications with DOI citations:

\begin{itemize}
\item Murphy et al. 2010: Pore size 100--200~\si{\micro\meter} for bone \cite{murphy2010}
\item Karageorgiou \& Kaplan 2005: Porosity $>$90\%, interconnectivity $>$90\% \cite{karageorgiou2005}
\item Gibson \& Ashby 1997: Mechanical scaling laws \cite{gibson1997}
\item Hildebrand \& R\"uegsegger 1997: Local thickness algorithm \cite{hildebrand1997}
\item Hulbert et al. 1970: Minimum pore size 100~\si{\micro\meter}
\item It\"al\"a et al. 2001: Optimal 300--400~\si{\micro\meter}
\item Hollister 2005: Scaffold design principles
\end{itemize}

The interactive ScaffoldEditor provides real-time validation feedback:
\begin{verbatim}
Pore size: 80 um < 100 um (FAIL)
  Target: 100-200 um (Murphy 2010, DOI:10.1016/...)
  Recommendation: Increase porogen size
\end{verbatim}

%% ============================================
\section{Illustrative examples}
%% ============================================

\subsection{TPMS scaffold generation with exact porosity}

Darwin generates analytical Triply Periodic Minimal Surfaces (Gyroid, Schwarz P, Schwarz D, Neovius) with exact porosity control:

\begin{lstlisting}[caption={Gyroid scaffold with 70\% target porosity}]
using Images, Statistics

size = 64
volume = zeros(Bool, size, size, size)

for i in 1:size, j in 1:size, k in 1:size
    x, y, z = 2pi .* (i, j, k) ./ size
    gyroid = sin(x)*cos(y) + sin(y)*cos(z) + sin(z)*cos(x)
    volume[i,j,k] = gyroid > 0.0  # threshold controls porosity
end

porosity = 1 - sum(volume) / length(volume)
# Output: 50.0% (exact match to analytical solution)
\end{lstlisting}

\subsection{Complete scaffold characterization}

\begin{lstlisting}[caption={Full morphometric analysis with ontology annotation}]
# Load and segment microCT data
volume = load_microct("scaffold.raw", (512,512,512))
binary = segment_otsu(volume)

# Compute metrics
metrics = compute_all_metrics(binary, voxel_size=10.0)
# porosity: 0.85, pore_size: 180 um, tau: 1.23, interconn: 0.98

# Validate against Q1 literature
validation = validate_for_tissue(:bone, metrics)
# All metrics within Murphy 2010 / Karageorgiou 2005 ranges

# Export with ontology annotations
export_jsonld("scaffold_fair.json", metrics,
              tissue=UBERON:0002481,  # bone
              material=CHEBI:53310)   # PCL
\end{lstlisting}

%% ============================================
\section{Validation}
%% ============================================

\subsection{Analytical ground truth (TPMS surfaces)}

Validation against TPMS surfaces with known analytical properties demonstrates algorithm correctness:

\begin{table}[h]
\centering
\caption{Validation against TPMS analytical ground truth (16 scaffolds: 4 types $\times$ 4 porosity levels)}
\label{tab:tpms}
\begin{tabular}{llll}
\toprule
Metric & Mean Error & Max Error & Status \\
\midrule
Porosity & 0.0\% & 0.0\% & PASS \\
Interconnectivity & 0.0\% & 0.0\% & PASS ($>$99\%) \\
Gibson-Ashby $E/E_s$ & 0.0\% & 0.0\% & PASS \\
\bottomrule
\end{tabular}
\end{table}

\subsection{Experimental ground truth (PoreScript dataset)}

Pore size was validated against the PoreScript dataset \cite{porescript}, which contains SEM images of salt-leached polymer scaffolds with manual measurements (n=374 pores across 3 images):

\begin{table}[h]
\centering
\caption{Validation against PoreScript manual measurements (DOI: 10.5281/zenodo.5562953)}
\label{tab:porescript}
\begin{tabular}{lllll}
\toprule
Sample & Darwin & Ground Truth & APE & Bias \\
\midrule
S1\_27x & 143.3~\si{\micro\meter} & 170.2~\si{\micro\meter} & 15.8\% & Under \\
S2\_27x & 152.2~\si{\micro\meter} & 175.7~\si{\micro\meter} & 13.4\% & Under \\
S3\_27x & 154.1~\si{\micro\meter} & 176.9~\si{\micro\meter} & 12.9\% & Under \\
\midrule
\textbf{Overall} & 149.4~\si{\micro\meter} & 174.0~\si{\micro\meter} & \textbf{14.1\%} & Under \\
\bottomrule
\end{tabular}
\end{table}

\subsection{Honest error analysis}

Darwin systematically underestimates pore size by approximately 15\% on 2D SEM images. This bias likely arises from:
\begin{itemize}
\item 2D sectioning effects (SEM captures cross-sections, not full 3D pore volumes)
\item Otsu thresholding sensitivity to image contrast
\item Connected component analysis merging adjacent pores
\end{itemize}

Users should apply appropriate correction factors when absolute pore size values are required. For comparative studies (e.g., scaffold A vs. scaffold B), the systematic bias cancels and relative comparisons remain valid.

The 14.1\% APE is comparable to the PoreScript algorithm's reported 15.5\% MAPE, though direct comparison is limited by different error metrics (APE vs. MAPE).

%% ============================================
\section{Impact}
%% ============================================

Darwin Scaffold Studio provides unique capabilities not available in existing tools:

\begin{enumerate}
\item \textbf{First public Q1 validation}: No other scaffold software publishes validation against analytical ground truth with 0\% error, plus honest experimental error analysis.

\item \textbf{Ontology-native design}: Integration of 1,200+ OBO Foundry terms enables machine-readable, FAIR-compliant scaffold characterization. Scaffolds can be queried by tissue type, cell compatibility, or material properties.

\item \textbf{Rigorous algorithms}: Local thickness and geometric tortuosity replace approximations with mathematically correct implementations.

\item \textbf{Cost accessibility}: Open-source alternative to commercial software (\$5,000--15,000/year), particularly relevant for research groups in developing countries.

\item \textbf{Reproducibility}: Self-contained examples run without external data. All validation scripts are included in the repository.
\end{enumerate}

The software is currently used at PUC-SP for bioactive glass scaffold characterization in bone tissue engineering research.

%% ============================================
\section{Conclusions}
%% ============================================

Darwin Scaffold Studio addresses critical gaps in tissue engineering scaffold analysis through validated algorithms, standardized ontology integration, and transparent error reporting. The platform achieves 0\% error on analytical TPMS ground truth and 14.1\% APE on experimental SEM measurements, with systematic bias documented for user awareness.

Future development includes:
\begin{itemize}
\item 3D microCT validation against BoneJ reference measurements
\item Machine learning segmentation for low-contrast images
\item Extended material library with degradation kinetics
\item Cloud-based analysis API
\end{itemize}

%% ============================================
\section*{Current code version}
%% ============================================

\begin{table}[h]
\centering
\begin{tabular}{ll}
\toprule
Code metadata & Description \\
\midrule
Current code version & v0.3.0 \\
Permanent link to code & \url{https://doi.org/10.5281/zenodo.17832882} \\
Code repository & \url{https://github.com/agourakis82/darwin-scaffold-studio} \\
Legal code license & MIT \\
Code versioning system & git \\
Software languages & Julia 1.10+ \\
Compilation requirements & None (interpreted) \\
Dependencies & Images.jl, Statistics.jl, XLSX.jl \\
Developer documentation & README.md, docs/, CLAUDE.md \\
Support email & demetrios@agourakis.med.br \\
\bottomrule
\end{tabular}
\end{table}

\section*{Acknowledgments}

The authors thank the PUC-SP Biomaterials and Regenerative Medicine Program. D.C.A. acknowledges the inspiration from the Demetrios language project for the epistemic type system design.

\section*{Declaration of competing interest}

The authors declare no competing interests.

\bibliographystyle{elsarticle-harv}
\begin{thebibliography}{10}

\bibitem[Murphy et al.(2010)]{murphy2010}
Murphy, C.M., Haugh, M.G., O'Brien, F.J., 2010. The effect of mean pore size on cell attachment, proliferation and migration in collagen-glycosaminoglycan scaffolds for bone tissue engineering. Biomaterials 31, 461--466. \url{https://doi.org/10.1016/j.biomaterials.2009.09.063}

\bibitem[Karageorgiou and Kaplan(2005)]{karageorgiou2005}
Karageorgiou, V., Kaplan, D., 2005. Porosity of 3D biomaterial scaffolds and osteogenesis. Biomaterials 26, 5474--5491. \url{https://doi.org/10.1016/j.biomaterials.2005.02.002}

\bibitem[Gibson and Ashby(1997)]{gibson1997}
Gibson, L.J., Ashby, M.F., 1997. Cellular Solids: Structure and Properties, 2nd ed. Cambridge University Press. ISBN: 978-0521499118.

\bibitem[Hildebrand and R\"uegsegger(1997)]{hildebrand1997}
Hildebrand, T., R\"uegsegger, P., 1997. A new method for the model-independent assessment of thickness in three-dimensional images. Journal of Microscopy 185, 67--75. \url{https://doi.org/10.1046/j.1365-2818.1997.1340694.x}

\bibitem[Jenkins et al.(2021)]{porescript}
Jenkins, M.J., et al., 2021. PoreScript: Automated Pore Size Analysis. Zenodo. \url{https://doi.org/10.5281/zenodo.5562953}

\bibitem[OBO Foundry(2024)]{obofoundry}
OBO Foundry. Open Biological and Biomedical Ontology Foundry. \url{https://obofoundry.org}

\bibitem[Hollister(2005)]{hollister2005}
Hollister, S.J., 2005. Porous scaffold design for tissue engineering. Nature Materials 4, 518--524. \url{https://doi.org/10.1038/nmat1421}

\end{thebibliography}

\end{document}
